\section{Overview}

The Robocup Logistic League is a competition to simulate Industry 4.0.  The goal of this contest is to use autonomous robots to fulfill some action by interacting with some industrial stations (MPS). Some constraints are applied like a defined number of MPSs, a defined size of the field, a limited amount of time for each phase of the game. 

\section{Changes in 2017}

The Smartbots team has participated in the Robocup German Open 2017 Magdeburg, Logistics League from 5. to 7. Mai 2017. Each year a version of the Rulebook is created. This year, it was published one month before the German Open. The first change of the year 2017 was a bigger field (14m x 8m) in comparison to the last year (12m x 6m). There were also more zones and each zone was smaller (1m x 1m) in comparison to last year (2m x 1.5m). The number of MPSs per team increased from 6 to 7 MPS with a new storage station. Before, possible zones with a MPS were sent by the Refbox to each team at the beginning of the exploration phase. In the rules 2017, there is no information sent. It is now necessary to explore the field and search for the 7 MPS and send the information, including the name of the MPS read with the help of the tag, the zone of the MPS and the orientation of MPS back to the Refbox. 

\begin{figure}%[tbhp]
\centering
\includegraphics[width=\linewidth]{pic/field.png}
\caption{Field for robocup 2017}
\label{fig:frog}
\end{figure}

\section{Difficulties}

Some difficulties have been faced because the Smartbots team noticed the new rules for the first time at the time of the Robocup. First, it was needed to implement a way to explore the field without previous knowledge about the position due to the change in the Rulebook. To face this, a path has been created with some fixed zones in the instruction planer component. Then, it was necessary to get the zone and the orientation of a detected MPS. This task has not been fulfilled for the Robocup. The last difficulty was to set up the network. There were two configurations, one to test and one to participate in the game.
 

\section{Situation in the Robocup}

To conclude, the team could accomplish some actions. First, only one robot was moving for each match because there was no multi-deployment. To do the exploration phase, the robot was going into some fixed positions (landmarks). There was no detection of the zone and the orientation of a MPS, although the Alvar tag detection was fully working. The production phase was not implemented and the maintenance phase was not handled. The communication between all components (Refbox Server, Instruction planer, Alvar Tag detection and MPS docking) was working and complete. 