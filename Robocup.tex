\section{Overview}

The Robocup Logistic League is a competition to simulate Industry 4.0.  The goal of this contest is to use autonomous robots to fulfill some action by interacting with some industrial stations (MPS). Some constraints are applied like a defined number of MPSs, a defined size of the field, a limited amount of time for each phase of the game.

\section{Changes in 2018}

The Smartbots@Ulm team will participate in the Robocup German Open 2018 in Magdeburg. Like the last two years the team is preparing for the logistic league. The changes in the Rulebook will be small. A new storage station will be  available on the field. More specific information are not available right now, but should be published to the end of February.

\begin{figure}%[tbhp]
\centering
\includegraphics[width=\linewidth]{pic/field.png}
\caption{Field for robocup 2017}
\label{fig:frog}
\end{figure}

\section{Lessons Learned of Robocup 2017}

In the year 2017 it was not clear how to setup the network for the game phases. The hosts of the Logistic League provides two configurations: one for testing and one for participating in the game. Therefoer it is recommended to speak with the hosts at the Robocup to set up the network properly. 

\section{Robocup Logistic League Participation}

This year it will be possible to play a complete Exploration Phase. The team will participate with one robot, which is able to detect stations, dock to stations and scan their ALVAR-tags. It will send the information back to the Referee Box and get some points. Maybe the team will configure the SmartInstructionPlanner component at the Robocup in a way, that three robots will participate at the game.
The production phase is not implemented yet, so the team will skip the third game phase.
