\subsection{Ideas for 2018}

The Robocup team 2017 have many ideas to improve the project for the next team. 

First ideas is that the exploration of the field should be efficient. In 2017, some fixed zones was explored. A first improvement can be to choose the nearest MPS station to approach. To do this solution, all MPS station already explorer should be store. A distributed database of detected stations can be a good solution even with multiple robotinos. To recognize the tag, a perfect docking should be working. In the last version, some imperfections was problematic. Another solution can be to use a smart agent which explore the field. The solution should be able to find each MPS of the team in the limited time.

Second point is to use the three robotinos in the field. Indeed, in 2017 only one robot was use. A multideployment is needed for this. A logic should be design to interact with all these robots. The last idea was to use one robot as a master and two robots as slave. A better solution can be to have a peer to peer implementation. Each robot can be a smart agent and communicate with the other.

Another idea was to test more the deployment before the contest. It is necessary to move robots before the competition in the laboratory. The problem is that the field in the laboratory is smaller that the real field in the competition. To face this, the Gazebo simulator should be used. With the simulator, a real situation can be tested. A previous implementation was done by the 2016 team. This approach is available in the git repository.

A idea to have a better code can be to implement more using SmartTCL. To do this, a lot of effort should be put in the lisp language. One or two persons of the team need to know very well this language. 

A important missing part during the contest was the production phase. Only the exploration phase was implemented in 2017. That's why the next task is to improve the exploration phase and to begin the production phase. It is important to be prepare to the future changes. Indeed, the contest become harder every year so a good idea can be to imagine the next modification in the rules. \\



//exploration of the field
//docking 
//choose the nearest MPS station to approach
//multideployment
//distributed database of detected stations with multiple robots X
//peer to peer implementation (master robot is not needed anymore) \\

//Use Gazebo simulator for testing (there were some approaches form the 2016 team. 
//Implement more using SmartTCL \\

//Start implementing parts of the production phase. \\

//Keep in mind contest becomes harder every year. 