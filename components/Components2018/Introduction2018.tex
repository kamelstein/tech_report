\section{Motivation}

The RoboCup is a master project of the master program "Information Systems" of the University of Applied Science Ulm. This project lasts two semesters, where the students have to organize and implement software regarding service robotics. The central goal of the project is to participate at the RoboCup German Open in Magdeburg. This includes not just the software implementation for the service robots, but also being organized and work collaboratively as a team with defined roles and activities, which include the organization of the participation of the RoboCup German Open, managing the residence, renting cars, signing insurances and handling the financial situations to successfully participate in the contest. 

The RoboCup German Open in Magdeburg is an international competition. 37 teams of well-respected Universities of 12 countries got together in Magdeburg 2017. Overall are six leagues presented at the RoboCup, like football, rescue missions, home robotics, industry and logistics. The team of the University of Applied Sciences Ulm is participating in the Logistics League. The topic of the league is the upcoming Industry 4.0. This year, it's the third participation of the team from Ulm in the Logistics League. It focuses on in-factory logistics applications to simulate a modern industrial working environment. The robots autonomously fulfill tasks and produce products. This domain is of enormeous interest since it covers the idea of the changing industry environment towards digitalization and Industry 4.0. 

\section{Objective}

This technical report provides the new team with information on the current state of the project. This includes the state of implementation of the components which will be used at the upcoming RoboCup 2018. New approaches are discussed (e.g. a new docking algorithm) and the architecture of the complete system is explained. The team's lessons learned are summarized as well as the difficulties during the project phase. In the end there is an outlook for upcoming tasks and the RoboCup 2019.
The report has the goal to make the project easier accessible for the new team members.  
