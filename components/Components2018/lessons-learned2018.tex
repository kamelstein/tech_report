This section describes the facts which all the team members learned through development of the Robocup software. It describes the organization of the
project team, what was wrong and what should be done better.

\section{Organization during the Semester}

\subsection{Project Handover from the previous Team}

When the team of 2018 took over the project from the 2017 seniors, Peter Franzreb, a former senior student of 2017 was available to answer questions and give detailed information about certain components. This was very helpful and is a must for upcoming handovers. \\

Each junior should not just look ahead shoulders and follow the implementations passively. They have to be actively integrated during the complete project phase. Therefore they need an own computer with the correct setup. \\

In the RoboCup project, lots of different repositories and versions are available and it is not clear, which is the right one to work with. Therefore every team should clean up their workspace for new members, which will continue at their workplace. \\

 Although there were documentation like the book of knowledge \cite{BOK} and the 2017 tech report available, it happened that some ideas or approaches were misunderstood from the new team members. Therefore a little introduction from the experienced staff of the laboratory is very important (regarding Sequencer, general architecture of SmartSoft, etc.). SmartSoft is a complex software and not easy to start with.\\

 Before the team gets to frustrated because of the complexity or just the not working tasks, it is highly recommended to ask the laboratory staff. There is not much time available between the Robocups to spend to much time (one or two weeks) on different kinds of issues.

\subsection{Meetings and assignments}

The team 2018 followed are sort of lightweight Scrum philosophy. Every working day (e.g. every Wednesday) a small meeting was scheduled. Therefore every member had an overview of what the other members are doing or working on. The team of 2018 consisted of three members, which would have been easy to handle without such meetings. But if the amount of team members increases, it is highly recommended to work in some agile environment with fix deadlines.

\subsection{Testing}

Testing was always integrated in the development process. The team tested the several components in isolation, and after they worked properly they got integrated and tested in the whole deployment. Thereby robust scenarios in the laboratory could be implemented, which should also work at the Robocup environment.


\section{Organization for the competition}

At the Robocup 2018 the team has a running scenario available, which will lead to a relaxed competition. The team can make small improvements to the software during the competition, but it is not necessary to implement or reimplement new components.

The team manager role is important. The team manager is responsible for renting cars and hotel booking. He act as bridge between the Robocup staff and the team. By doing so, the other team members can focus on development and testing of the software.
