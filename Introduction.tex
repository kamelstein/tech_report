\subsection{Motivation}

In the University of Applied Science Ulms' masters program Informationsystems each student gets hand on experience in developing and implementing software in a group project by participating in a two semester lasting project. One of the projects is the robotics project in the service robotic research laboratory (zafh) of the University of Applied Science Ulm headed by Prof. Dr. Christian Schlegel. The service robotic research laboratory is focused on research in the field of methods, algorithms and software tools for implementing service robots and autonomous systems for everyday use. \\
For several years students of the masters course participated in the robotics project. The project is focused on industrial robots and the students gather theoretical and practical experience in working, developing and implementing software for service robots. Each project includes the participation of the RoboCup German Open (RGO) where students from universities all over the world come together and compete against each other in different contests. The SmartBots@Ulm, the project team of the masters course Informationsystems of the University of Applied Science Ulm participated in the RoboCup German Open 2017 in the RoboCup Logistics League (RCLL). It focuses on in-factory logistical applications to simulate a modern industrial working environment. The robots autonomously fulfill tasks and produce working output. This section is of enormous interest since it covers the idea of the changing industry environment towards digitalisation and industry 4.0 requirements. \\
The project not only included gaining experience in developing and implementing software for service robots but also gaining the experience of working in a long term group project and organizing the project self-standingly. The group projects task were not only developing software and putting it together but also being organized and work collaboratively as a team with defined roles and activities, which included the organization of the participation of the RoboCup German Open, managing the residence, renting cars, signing insurances and handling the financial situations to successfully participate in the contest. 

\subsection{Objective of the Paper}

This technical report funds its reader with the basic understanding of the current state of the software, components and deployments of the robotinos used in the RoboCup German Open 2017. At first, there is a small overview of the principles of the RoboCup German Open 2017 and the difficulities the teams had to face. The next chapter is giving insights on the current state of the used components and the way they work and interact together. All the components are explained and the way they have been adjusted to the last teams version and what difficulties have been faced. The teams lessons learned are stated in the subsequent chapter and the last chapter is about the ideas and open topics for next years student team